\documentclass[a4paper]{article}



%Packages included
\usepackage{graphicx}
\usepackage{multicol}
\usepackage{hyperref}
\usepackage[headings]{fullpage}
\usepackage{fancyhdr}
\usepackage{float}
\usepackage{caption}
\usepackage{subcaption}
\usepackage[
backend=bibtex,
citestyle=authoryear,
sorting=none
]{biblatex}
\addbibresource{../bibliography/refrencedpapers.bib}

%set graphics path
\graphicspath{{../plotting/figures/}}

%opening
\title{The Effect of Bathymetry Changes on Meridional Overturning Currents}
\author{Marte Voorneveld\\
		5911591}


\pagestyle{fancy}
\fancyhead{}
\fancyfoot{}
\fancyhead[C]{\textit{The Effect of Bathymetry Changes on Overturning Circulation}}
\fancyhead[L]{\thepage}
\renewcommand{\headrulewidth}{0pt}

\begin{document}

\maketitle
\noindent\rule{\textwidth}{1pt}
\begin{abstract}

\end{abstract}
\noindent\rule{\textwidth}{1pt}
\begin{multicols}{2}
\section{Introduction}
The geometry and resulting bathymetry of our planet is an ever changing phenomenon. In the last 120 Ma the earth moved from having one major oceanic system in the Pacific with a single large continent to the current 3 ocean system (\cite{besse2002apparent}). The Bathymetry changes that occurred in this time period are characterized by the opening and closing of certain passages through which exchange of water between the oceanic basins is observed. The exact timing of passage openings is a topic of rigorous debate in literature (\cite{Scher2006Apr}, \cite{Schmidt2007Jan}).


One of the changes on which there is general consensus, is the inception and expansion of the Atlantic ocean and the resulting decrease in size of the Pacific basin. The creation of the Atlantic basin has had major effects on the earth's climate, especially resulting in massive localized changes such as the temperate European climate, due to the north Atlantic meridional overturning current (AMOC). This creates the current Northern sinking oceanic throughflow in the Atlantic. However it is unknown when exactly this northern sinking started. With the past non-existance of the Atlantic it must have started some time in the last 40Ma with the advent of a larger Atlantic (\cite{Abelson2017onset}). 

The result of these bathymetry changes on the oceanic stream function and the resulting overturning currents is something that has been previously studied by \cite{Mulder2017Jul}. They however found that using a Jacobian matrix for continuation in each of the model years fails to simulate the onset of the Northern sinking AMOC that is physically observed. Here we instead propose to use a general circulation ocean model with only a changing bathymetry keeping the same initial forcing for each time step. This eliminates the need for a continuation using the Jacobian matrix method proposed in \cite{Mulder2017Jul}.

This paper will focus solely on changes in bathymetry using simplified zonally averaged global forcings. The results of the model will be used to estimate global changes in oceanic through flow at the critical passages. Furthermore the strength of the meridional overturning currents (MOC) will be studied.


\section{Methods}



\subsection{Simplified Forcings}
The domain chosen is bounded by longitudes $\phi_E=-180^{\circ}$ and $\phi_W=-180^{\circ}$ and latitudes $\theta_N=80^{\circ}$ and $\theta_S=-80^{\circ}$ with periodic boundary conditions in the zonal direction.
Furthermore the model uses restoring boundary conditions first proposed by \cite{haney1971surface}. Restoring the boundary at the surface of the oceanic basin to be a certain value based of a forcing field for Sea Surface Temperature (SST), Sea Surface Salinity (SSS), and wind stresses ($\tau$).
 The depth profile has 15 layers with grid stretching. There are $90 \times 40$ grid points to make a $4^{\circ} \times 4^{\circ}$ resolution model. The forcings are prescribed as in \cite{Mulder2017Jul} by highly idealized zonally averaged forcings using current day values. The SST and Zonal wind stress are chosen as the analytical model in \cite{bryan1987parameter}. While SSS is chosen as a zonal average of current day values (from ECMWF but not sure how to cite). The meridional wind stress is set to zero. The maximum ocean depth is 5000m. The model also requires initial conditions for salinity and temperature, for this zonally averaged present day values are again used.

\end{multicols}
%example full width overturning

\begin{figure}[H]
\begin{subfigure}{.5\textwidth}
	\includegraphics[width=0.9\linewidth]{sss_sst_profile.png}
	\caption{SSS and SST}
	\label{fig:sst_sss}
\end{subfigure}
\begin{subfigure}{.5\textwidth}
	\centering
	% include first image
	\includegraphics[width=0.9\linewidth]{tau_x_profile.png}
	\caption{Zonal wind stress ($\tau_x$) profile}
	\label{fig:tau_X}
\end{subfigure}

\caption{Idealized forcing profiles}

\end{figure}

\begin{multicols}{2}
\subsection{Creating Bathymetries}
Creating the bathymetries for the model was done using bathymetries from Muller et al.\cite{Muller2008Mar} these were scaled to a 4 degree model and subsequently changed to address passage openings in the 4 degree case where, due to the low resolution of the model, choices have to be made with respect to the opening of certain passages. One of the choices that was made specifically is to change the bathymetry of the standard 4 degree model to a custom one made in the same process as the other bathymetries too more accurately portray changes that occur using this process. One of the choices that was made is that the northern Sea is closed of in all of the bathymetries. This is mainly due to the fact that 4 degree models do not have enough resolution to support This sea and can cause strange behaviour to occur. Also there is little connection to the other oceans, thus negating the need for such a basin to be in our model.

The main events that shape the oceanic passages can be devided into time periods. These time periods are defined as follows in this paper. This deviates from their definitions in literature but serves only as a means of applying a name to the time steps.

\begin{table}[H]
	\begin{tabular}{lll}
		&From &Until \\
		Paleocene & 65Ma&55Ma    \\
		Eocene    & 50Ma&35Ma     \\
		Oligocene & 30Ma&20Ma    \\
		Miocene   & 20Ma&Present 
	\end{tabular}
\end{table}

%TODO (Relevant figures from paleoscene showing changes)
The model is started in the Paleocene where in the beginning a vast Pacific exists almost serving as a single basin. This period is largely characterized by the growth and development of a larger atlantic basin. Serving to decrease the size of the pacific basin. This can be seen in Figure %TODO add figures showing diffirence in size. 
Where the changes during the paleoscene are shown.
\begin{figure}[H]
	
	
	\centering
	\begin{subfigure}[b]{\linewidth}
		\centering
		\includegraphics[width=\linewidth]{bathymetry/baath_65.png}
		\caption{Beginning of the Paleocene}
	\end{subfigure}
\begin{subfigure}[b]{\linewidth}
	\centering
	\includegraphics[width=\linewidth]{bathymetry/baath_50.png}
	\caption{End of the Paleocene}
\end{subfigure}
	\caption{}
	\label{fig:paleocene_bath}
\end{figure}


%TODO Eocene Drake-indian-tasman passages + depth

The Eocene in contrast to the Paleocene is destinguised by The opening of certain passages connecting oceanic basins. These effects are often studied extensively for each individual passage. Choosing the exact timespan for opening the passages is done manualy by looking at often active research takin into account big uncertaincies in the exact timing of the openings. The Eocene is characterized by several large events shown in Figure %todo add figure showing the events in detail
The first of such events that occurs is the indian continent colliding with the eurasian continent This has the effect of closing the deep water formations between the Thetys sea and the Indian ocean.
Next the Tasman passage is opened\cite{Lawver2003Sep} as a shallow passage slowly growing in size. The tasman passage opening is believed to have had a large impact on the onset of the ACC. The Total circulation of water around the antartic basin is finalized by the opening of the shallow Drake passage some 30Ma. 30Ma is specifically chosen to diffirentiate between the diffirent passage openings. Especially since there is still ongoing debate on the exact timing of drake passage opening.
%30ma drake passage opening (chosen as to diffirentiate between diffirent stages)
%35ma Indian passage closure
%35ma Tasman passage opening
%Indian continent colliding

%TODO Oligocene changes
The next time period is the oligocene Which is largely characterized by the deepening of the Tasman and drake passage and further expansion of the atlantic basin. It is believed %source
That the onset of the northern sinking atlantic started in this time period. Also the ACC probably is increasing in strenght.


%TODO Miocene

The miocene is Characterized by yet more increase in strenght of the ACC. Also Some more passage closures occur. Starting with the closure of the Thetys Seaway which had been decreasing in size in the previous 20Ma. Also the passage between modern day australia and indonesia is significantly decreasing in size due to the onset  of multiple vulcanic islands making the passage more narrow and shallow. Showing especially in this model. Further more the Thethys gateway
%0Ma current
%5Ma Wider Indonesia
%10ma last CAS
%14ma Thetys seaway last open 15ma first occurrence



Drake passage opening \cite{Scher2006Apr}

Tasman passage opening 

Central American seaway closure \cite{Molnar2008Jun} (deepwater 7Ma (10 for paper))\cite{Pindell1988Dec}

Tethys seaway closure \cite{Hamon2013Nov}

Widening of indonsian seaway (due to australia moving up.)


titis passage opening 

Choises made. 

Examples of where these choises interfere with reality.

Individual passages.

What age are we dealing with (Events/Changes observed in literature)


\end{multicols}
%example full width overturning

\begin{figure}[H]
\begin{subfigure}{.5\textwidth}
	\includegraphics[width=0.9\linewidth]{bathymetry/baath_65.png}
%	\caption{Beginning of the Paleocene}
	\label{fig:paleo_begin}
\end{subfigure}
\begin{subfigure}{.5\textwidth}
	\centering
	% include first image
	\includegraphics[width=0.9\linewidth]{bathymetry/baath_50.png}
%	\caption{Beginning of the Eocene}
	\label{fig:paleo_end}
\end{subfigure}
\begin{subfigure}{.5\textwidth}
	\centering
	% include first image
	\includegraphics[width=0.9\linewidth]{bathymetry/baath_30.png}
%	\caption{Beginning of the Oligocene}
	\label{fig:paleo_end}
\end{subfigure}
\begin{subfigure}{.5\textwidth}
	\centering
	% include first image
	\includegraphics[width=0.9\linewidth]{bathymetry/baath_20.png}
%	\caption{Beginning of the Miocene}
	\label{fig:paleo_end}
\end{subfigure}
\begin{subfigure}{.5\textwidth}
	\centering
	% include first image
	\includegraphics[width=0.9\linewidth]{bathymetry/baath_0.png}
%	\caption{Present day}
	\label{fig:paleo_end}
\end{subfigure}
\caption{Paleocene Begin and End}
\label{fig:paleocene_bath}

\end{figure}

\begin{multicols}{2}




\subsection{Veros and Runtime}
\subsection{Veros}
Ocean modeling has been an area of continued progress. The resolutions of the models have been steadily increasing since the inception of the first computerized ocean models. However, due to the age of some of these models and the continued adaptation of often old legacy Fortran code, many models have become enormous hurdles to get started with often resulting in frustration. The Veros ocean model project is trying to tackle this problem with a totally new code base written entirely in Python (\cite{Hafner2018Aug}). 
Veros is an ocean general circulation model (GCM) based on the successful PyOm2 model. It was designed from the ground up with flexibility in mind. This flexibility cuts valuable time spent on figuring out the often cumbersome Fortran models of the past. Veros is specifically well suited for researching the effect of changes in both forcings and bathymetrys (depth profiles of the oceans). They can be easily edited using Python. These features in particular are heavily used in this paper. One of the most extensively used features for example is the fact that any bathymetry can, without further manual specifications, be used for stream function calculation.
The fact that it is fully written in Python is especially useful as python is far more widespread than Fortran and it is thus much easier to teach Veros to new students. 

In this case the models used in this paper are run on an 8 core (16 threads) machine using an MPI CPU configuration of 1 node. This is sufficient for the lower resolution models used in this paper. But Veros allows the usage of multiple nodes to do calculations on much higher resolution problems.



\section{Results}
\subsection{Stabilizing of the models}
(Section on when the integration was stopped. How good it is etc.)
$60/100$ done


\subsection{Passage throughflow}
(Section on throughflow for each passage)
$50/100$ done

\subsection{Stream function}

(Section on the Long-Lat stream function and vertical zonnally integrated streamfunction)
$60/100$ done
\section{Summary}
%\section{Summary}
In this paper we have presented a simplified approach to the modeling of past climate systems using Veros. This paper focused heavily on simplified forcings of the global oceanic basins. This allows in contrast to past attempts using continuation approaches to be able to efficiently look at the effect of changes in geometry on the major oceanic flows. The results shown here are of relatively low resolution and highly idealized boundary conditions. But they still manage to capture many of the effects seen in more complex coupled models. The integrations were done on a consumer computer showing that it is now possible to do big ocean simulation research on readily availible hardware.

The analysis of the results focused specifically on the multiple passage changes that occured in the time period. Specifically noting the effects on the thermohaline circulation.


$0/100$ done

\section{Discussion}
%The method presented here for an alternative to the continuation approach suggested by \cite{Mulder2017Jul} still has quite a few problems. First of all, the $4^{\circ}$ resolution together with the limited number of depth layers is one of the main problems. The limited depth layers fail to accurately capture even the present day overturning circulation. Here we note that the same can be said for a model with present day forcings as noted in \cref{sec:mocQual}.

The possibility of using a 1 or 2 degree Veros model for this paper was extensively explored. But issues often arose with the exact values of constants and frequent invalid value errors to do with eddy kinetic energy could not be fixed in time for this paper.

The 4 degree model also took quite a bit of time to be adapted for the customized forcings and bathymetries. This is due to the fact that the method for determining boundary conditions for islands would often find more islands than exist in the model. Resulting in having to customize each setup individually for its bathymetry to accept the islands present. This is also why $180^{\circ} E$ was used as the boundary longitude instead of the default $0^{\circ}$. These problems are one of the main driving forces behind the decision to limit the integration time to just 500 years. A better estimate for the overturning circulation can probably be made already when extending the integration time to 1000 years

Another major change that is neglected in this paper is the total absence of change in surface forcings over time. Even though it is known that these change drastically even in short time spans. We also have some general knowledge of global average temperature for the time period discussed here. However the search for a dataset for each time step has proven futile. Often large uncertainties exist which would result in much more confusing results. This left us with the decision to not bother with any changes in the forcings.

A lot of future research is possible in the topic of oceanic throughflow. More accurate bathymetries are being produced due to breakthroughs in geological techniques (\cite{Baatsen2016Aug}). These, coupled with a higher resolution model will probably result in even more accurate depictions of the past oceanic systems. Research on this topic is of particular importance because of the present day observed changes in strength of the MOC and to enhance the feedback loop between what is observed in geological excavations and models of our planet.
$0/100$ done


%\section{Test images}
%\begin{figure}[H]
%	\includegraphics[width=\linewidth]{overturning_overview.png}
%	\caption{test caption}
%	\label{fig:example1}
%\end{figure}
%\end{multicols}
%%example full width overturning
%\begin{figure}[H]
%	\includegraphics[width=\linewidth]{overturning_overview.png}
%	\caption{test caption}
%	\label{fig:example1}
%\end{figure}
%
%\begin{multicols}{2}

\printbibliography

\end{multicols}



\end{document}
