\documentclass[a4paper]{article}

%Packages included

\usepackage{multicol}
\usepackage[headings]{fullpage}
\usepackage{fancyhdr}
\usepackage[
backend=bibtex,
]{biblatex}
\addbibresource{../bibliography/refrencedpapers.bib}


%opening
\title{Very long term change in oceanic through flow}
\author{M.D. Voorneveld}


\pagestyle{fancy}
\fancyhead{}
\fancyfoot{}
\fancyhead[C]{title}
\fancyhead[L]{\thepage}
\renewcommand{\headrulewidth}{0pt}

\begin{document}

\maketitle
\noindent\rule{\textwidth}{1pt}
\begin{abstract}

\end{abstract}
\noindent\rule{\textwidth}{1pt}
\begin{multicols}{2}
\section{Introduction}

In the past 65 Ma of earth's history major changes have occurred in the climate. One of the changes that has been extensively studied is the change in geometry of the earth\cite{besse2002apparent}. In this period the earth moved from having one major oceanic system in the Pacific with a single large continent to the current 3 ocean system. The Bathymetry changes that occured in this time period are characterized by the opening and closing of certain passages through which exchange of water between the oceanic basins is determined. The exact timing of passage openings is a topic of rigorous debate\cite{Scher2006Apr}\cite{Schmidt2007Jan}.


One of the changes on which there is general consensus is the inception and expansion of the Atlantic ocean and the resulting decrease in size of the Pacific basin. The inception of the Atlantic has had major effects on the earth's climate especially resulting in massive localized changes such as the temperate European climate due to the north Atlantic meridional overturning current.


Ocean modeling has been an area of continued progress. The resolutions of the models has been increasing steadily since the inception of the first digital ocean models. However, due to the age of some of these models and the continued adaptation of often old legacy Fortran code, the older models have become enormous hurdles to get started with. The Veros \cite{Hafner2018Aug} ocean model project is trying to tackle this problem with a totally new code base written entirely in Python. Veros allows easy editing of forcing and geometry input and infinite flexibility in the model's setup.

Due to the change in the Pacific basin specifically it is very interesting to look at the occurrence of a northern sinking solution similar to the Atlantic we are accustomed to today.
This paper will focus solely on changes in bathymetry using very simplified zonally averaged global forcings to estimate global changes in oceanic through flow and strength of the meridional overturning currents (MOC).




\section{Methods}
\subsection{On veros}
\subsection{Simplified Forcings}
\subsection{Scaling oceanic basins}
\subsection{Measuring overturning currents}
About the overturning currents

converting the data from \cite{Muller2008Mar} to veros

ignoring many things
idealized global temprature profile
idealized global salinity
idealized global wind stress


\section{Results}

\section{Summary}

\printbibliography

\end{multicols}



\end{document}
