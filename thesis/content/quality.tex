\subsection{Model Quality}
To asses the quality of the model and thus if any of the results resemble reality, one can compare the present day setup of the model to an existing model with realistic forcings. Also, the model can be compared to a higher resolution model. This is done to check how good the model is and whether any of the results have any connection to reality.

\subsubsection{Quality of BSF}
Too look at the quality of the barotropic stream function we compare the barotropic stream function of our model to a $4^{\circ}$ model with realistic forcings. This model was made with the standard Veros setup with custom open Indonesian passage. In \fref{fig:bsf_compared} we see that the barotropic stream function itself is remarkably similar. Only showing a weaker subtropical gyre in the Indian ocean and a weaker gulf stream. In the temperature profile at 245 meters in depth there are however larger discrepancies. There is a $4^{\circ}C$ difference in temperature in the gulf stream and a $2^{\circ}C$ temperature difference in the Kuroshio gyre. This can be attributed to weaker SST forcing at the surface on these places because of the zonal mean nature of these forcings and also due to the shorter runlength of the model. Thus the model did not yet get addequate time to fully develop the flows (THIS COULD BE BETTER)

