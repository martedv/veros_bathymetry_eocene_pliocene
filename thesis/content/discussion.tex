The method presented here for an alternative to the continuation approach suggested by \cite{Mulder2017Jul} still has quite a few problems. First of all, the $4^{\circ}$ resolution together with the limited number of depth layers is one of the main problems. The limited depth layers fail to accurately capture even the present day overturning circulation. Here we note that the same can be said for a model with present day forcings as noted in \cref{sec:mocQual}.

The possibility of using a 1 or 2 degree Veros model for this paper was extensively explored. But issues often arose with the exact values of constants and frequent invalid value errors to do with eddy kinetic energy could not be fixed in time for this paper.

The 4 degree model also took quite a bit of time to be adapted for the customized forcings and bathymetries. This is due to the fact that the method for determining boundary conditions for islands would often find more islands than exist in the model. Resulting in having to customize each setup individually for its bathymetry to accept the islands present. This is also why $180^{\circ} E$ was used as the boundary longitude instead of the default $0^{\circ}$. These problems are one of the main driving forces behind the decision to limit the integration time to just 500 years. A better estimate for the overturning circulation can probably be made already when extending the integration time to 1000 years

Another major change that is neglected in this paper is the total absence of change in surface forcings over time. Even though it is known that these change drastically even in short time spans. We also have some general knowledge of global average temperature for the time period discussed here. However the search for a dataset for each time step has proven futile. Often large uncertainties exist which would result in much more confusing results. This left us with the decision to not bother with any changes in the forcings.

A lot of future research is possible in the topic of oceanic throughflow. More accurate bathymetries are being produced due to breakthroughs in geological techniques (\cite{Baatsen2016Aug}). These, coupled with a higher resolution model will probably result in even more accurate depictions of the past oceanic systems. Research on this topic is of particular importance because of the present day observed changes in strength of the MOC and to enhance the feedback loop between what is observed in geological excavations and models of our planet.