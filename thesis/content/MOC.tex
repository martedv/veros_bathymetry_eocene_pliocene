\subsection{MOC Stream function}

Next we will do an analysis on the Global Meridional overturning current stream function (MOC). As mentioned in \cref{sec:MOCSTREAM} these values will probably not result in a very realistic picture of the overturning circulation. However, it still gives us a rough general idea of the deep ocean flows of the thermohaline circulation. The main focus in this section will be on wind driven circulations as they seem to be quite accurately captured as discussed in \cref{sec:mocQual}. 

The MOC stream function we observe little variability until the onset of the "ACC-like" cell at 35Ma (\cref{fig:moc_total}). However, we do observe some changes in the period extending to the late Eocene. In the Paleocene one of the most interesting features seen in the MOC stream function is a southern cell extending from the equator to the Antarctic continent and a mirrored cell in the north. Almost mirroring around the equator. However the southern cell is observed to be stronger with a weaker  A strong ($9 Sv$) southern cell exists until 55 Ma where its extend is not as far south.

\subsubsection{Eocene}\label{sec:eocenemoc}
%50-35
In the Eocene, we observe little difference to the stream function in the Paleocene. The most interesting feature is the southern subpolar grye. Which starts to extend downwards like the ACC in the present day. The antarctic bottomwater cell is subsequently significantly reduced in strength.
The origin of this "ACC-like" cell at 35Ma is attributed to the opening of the Tasman passage in the bathymetry. This is similar to the open Tasman and closed Drake passage case shown by \cite{Sijp2011Dec}.  

\subsubsection{Oligocene}
%30-20
The Oligocene is characterized by the strong onset of the ACC and a subsequent decrease in size of the south polar cell. It being "pushed" aside due to the strong ACC currents. Another interesting artifact of this is that in the $30 Ma$ setup a kind of overturning current is observed. This is however not shown in the 25 and 20 Ma time steps. It should be noted that the deep water cells (<-2000 m) in all of these are still too weak to be of any realistic value. The Oligocene does seem to harbour some of the strongest Polar cells in any of the models. This was not necicarily observed in the Pictures of the BSF.

\subsubsection{Miocene}
%20-0

The Miocene shows some of the main features of the MOC. One of these is the overturning current previously mentioned. It is however important to note that it is still really weak compared to other models with more depth layers and observations (\cite{von2006effect}). Probably due to the fact that the overturning circulation in the Atlantic is absent in this model. This may simply be a case of boundary conditions but could also be explained by the relatively weak SSS forcings in the Atlantic compared to real world values. In the last 15 Ma we see very little change overall in the MOC stream function. We see mostly fluctuations in the northern sub polar cell. 