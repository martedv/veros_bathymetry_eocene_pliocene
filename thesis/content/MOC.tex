Next we will do an analysis on the Global Meridional overturning current stream function (MOC). As mentioned in \prettyref{sec:MOCSTREAM} these values will probably not result in a very realistic picture of the overturning circulation. However it still gives us a rough general idea of the deep ocean flows of the thermohaline circulation. This section will again be split into the time periods as defined in \prettyref{sec:bathys}.

\subsubsection{Paleocene}
%65-55
In the Paleocene one of the most interesting aspect is the southern cell extending from the equator to the antarctic continent. A strong ($9 Sv$) southern cell exists. This cell is largely responsible for the mostly positive nature of the overturning circulation seen in the BSF.

(need to research more)


\subsubsection{Eocene}
%50-35
In the eocene we observe little diffirence to the streamfunction in the paleocene. The most interesting feature is the onset of the previously mentioned "proto ACC" in which the southern cell extends downwards. This indicates that the proto ACC might be captured by the model.


\subsubsection{Oligocene}
%30-20
The oligocene is characterized by the strong onset of the ACC and a subsequent decrease in size of the south polar cell. It being "pushed" aside due to the strong ACC currents. Another interesting artifact of this is that in the $30 Ma$ setup a kind of overturning current is observed. This is however not shown in the 25 and 20 Ma time steps. It should be noted that the deep water cells (<-2000 m) in all of these are still too weak to be of any realistic value. The Oligocene does seem to harbour some of the strongest Polar cells in any of the models. This was not necicarily observed in the Pictures of the BSF.

\subsubsection{Miocene}
%20-0

The miocene shows some of the main features of the MOC. One of these is the overturning current previously mentioned. It is however important to note that it is still really weak compared to other models with more depth layers and observations (\cite{von2006effect}). Probably due to the fact that the overturning circulation in the Atlantic is absent in this model. This may simply be a case of boundary conditions but could also be explained by the relatively weak SSS forcings in the Atlantic compared to real world values. In the last 15 Ma we see very little change overall in the MOC stream function. We see mostly fluctuations in the northern sub polar cell. 