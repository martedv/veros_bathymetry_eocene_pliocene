\subsection{Creating Bathymetries} \label{sec:bathys}
To facilitate the model a set of 14 bathymetries was created in $5$Ma time steps. These run from 65Ma to the present-day configuration. These were reconstructed from bathymetries gained in \cite{Baatsen2016Aug}.
These bathymetries which originally were $0.5^{\circ} \times0.5^{\circ}$ have been scaled to a $4^{\circ} \times 4^{\circ}$  model using Gaussian interpolation. Next, the land masks were manually edited to include passages and exclude some inland seas. Due to the low resolution of the model, choices have to be made concerning the opening of certain passages. One of the choices that were made is that the northern Arctic sea is closed off in all of the bathymetries. This is mainly since $4^{\circ} \times 4^{\circ}$ models do not have enough resolution to facilitate this sea and Veros lacking the ability to have polar flow.
The main events that shape the oceanic passages can be divided into periods. These periods are defined in \cref{tab:timeperiods}.
\begin{table}[H]
\centering
	\begin{tabular}{lll}
		&From &Until \\
		Paleocene & 65Ma&55Ma    \\
		Eocene    & 50Ma&35Ma     \\
		Oligocene & 30Ma&25Ma    \\
		Miocene   & 20Ma & 10Ma  \\
		Pliocene & 5Ma & present
	\end{tabular}
\caption{Time periods covered by this paper}
\label{tab:timeperiods}
\end{table}
The discussion on each period is split into separate subsections. Here we address each of the periods and their respective changes.

\subsubsection{Paleocene}
In the Paleocene a vast Pacific exists almost serving as a single basin. This period is largely characterized by the growth and development of a larger atlantic basin. Subsequently a decrease in size of the pacific basin is also observed. The drake passage is explicitly chosen to be closed in this time period, there is some evidence of it being opened in the paleocene due to a major change in the motion of the South American and Antarctic plates until about 50Ma (\cite{Livermore2005Jul}). However, the evidence proposes a shallow opening of less than 1 km in depth. These uncertainties and the shallow nature of the opening has led to the decision to close the passage until its certain deep water connection starting after the late Eocene as also indicated by \cite{Livermore2005Jul}.

It is also interesting to note the existence of a range of islands between the Indian continent and the Eurasian continent which dissapears in this period.  These islands are called the Kohistan-Ladakh Arc (\cite{Jagoutz2009}). These may have had quite significant effect on the flow through the thetys seaway and are thus an interesting topic to discuss later on. 
\subsubsection{Eocene}
The Eocene in contrast to the Paleocene is distinguished by the opening of certain passages connecting oceanic basins. These effects are often studied extensively for each individual passage in literature. Choosing the exact timing for opening the passages is done manually by looking at often active research. The first of such passages to open is the Tasman passage which is opened at 35Ma as a shallow passage slowly growing in size(\cite{Lawver2003Sep}). The Tasman passage opening is believed to have had a large impact on the onset of the Atlantic circumpolar current (ACC). Some authors even suggests its influence on the onset of a early "proto-ACC" (\cite{Sarkar2019Jul}). This proto-ACC may have caused upwelling of northern-sourced nutrient-rich deep equatorial Pacific waters in the south Pacific. However, this is assuming an open drake passage, which does not exist in our bathymetries. Thus, this upwelling will likely not be observed until the early Oligocene. 

From the onset of the early Eogene the Indian Continent has been fast moving towards the north slowly closing the northern passage between the Indian ocean and the Tethys seaway. The deep water passage is closed from 35Ma based onwards \cite{Najman2010Dec}. This limits the throughflow through the Thetys seaway to purely east of the Indian continent. Which is now in effect part of the larger Eurasian continent. 
\end{multicols}
%example full width overturning

\begin{figure}[H]
\includegraphics[width=\linewidth]{bathymetries_paleocene_crop.pdf}
\caption{Paleocene bathymetries, black squares show the land mask used. The bathymetries for \textbf{(a)} 65 Ma including the Kohistan-Ladakh Arc \textbf{(b)} 60 Ma showing submergal of the Kohistan-Ladakh Arc \textbf{(c)} 55Ma here the gap between africa and south America is seen to grow.}
\end{figure}

\begin{figure}[H]
	\includegraphics[width=\linewidth]{bathymetries_eocene_crop.pdf}
	\caption{Eocene bathymetries, black squares show the land mask used. The bathymetries for \textbf{(a)} 50 Ma. \textbf{(b)} 45 Ma. \textbf{(c)} 40 Ma.\textbf{(d)} 35 Ma showing the opening of the Tasman passage by the detachment of the Australian continent from Antarctica. Also, water passage over India is closed.}
\end{figure}

\begin{multicols}{2}


\subsubsection{Oligocene}
%30-25
From the onset of the Oligocene, the total circulation of water around the Antarctic basin is finalized by the opening of the shallow Drake passage at around 30Ma. 30Ma is specifically chosen to differentiate between the opening of the drake and Tasman passages. Especially since there is still some debate on the exact timing of drake passage opening (\cite{Scher2006Apr}; \cite{Livermore2005Jul}). These openings coincide with the onset of the ACC that has had major effects on the global climate variability. Furthermore, The Oligocene is characterized by the further expansion of the Atlantic basin and a shallower Thetys seaway. Furthermore, a deep water area starts existing between what is now Greenland and the European continent. This water basin is now known to be central to the deepwater formations of the northern Atlantic.
\end{multicols}
%example full width overturning

\begin{figure}[H]
\includegraphics[width=\linewidth]{bathymetries_oligocene_crop.pdf}
\caption{Oligocene bathymetries, black squares show the land mask used. The bathymetries for \textbf{(a)} 30 Ma here the Drake passage is opened, this detaches south America from Antarctica and allows the onset of the ACC  \textbf{(b)} 25 Ma where the Thetys Seaway is seen to be slowly shrinking.}
\end{figure}
\begin{multicols}{2}
\subsubsection{Miocene}
%10-15-20

The Miocene is characterized by the slow closure of the Thetys seaway. It had been slowly decreasing in size since the beginning of the Eocene and finally fully detaches the northern passage of flow between the Mediterranean and the Indian ocean from 15Ma onward(\cite{Hamon2013Nov}). Another feature that is not captured particularly well by this model but should definitively be mentioned, is the decrease in the size of the Indonesian passage. Which is mainly due to the onset of the volcanic islands we know today.

\subsubsection{Pliocene}
% 5-0
In the Pliocene the panama passage is finally closed off (\cite{Molnar2008Jun}; \cite{Pindell1988Dec}). This coincides with the present-day situation. Due to the closure the mid latitude throughflow between the Atlantic and Pacific basins is believed to have started. The throughflow in the panama seaway is believed to have reversed in direction with the onset of the decrease in size and subsequent closure of the Thetys seaway (\cite{von2006effect}; \cite{omta2003physical}). Something that will be studied more closely in the discussion of our results. Also it is of note that the present day bathymetry used here was made using the same method as the other bathymetries. This decision was made over reusing the existing bathymetry used by the standard 4-degree model. This was done to make a better discussion possible. 
\end{multicols}
\begin{figure}[H]
	\includegraphics[width=\linewidth]{bathymetries_miocene_crop.pdf}
	\caption{Miocene bathymetries, black squares show the land mask used. The bathymetries for \textbf{(a)} 20 Ma. \textbf{(b)} 15 Ma Here, the closure of the Thetys Seaway is seen which connects the African continent to the Eurasian continent and stops the northern flow of water between the Indian and Atlantic Oceans. \textbf{(c)} 10 Ma Here the panama seaway is seen to become much more shallow. Also the Indonesian passage has shrunk drastically in this period.}
\end{figure}
\begin{figure}[H]
	\includegraphics[width=\linewidth]{bathymetries_pliocene_crop.pdf}
	\caption{Pliocene bathymetries, black squares show the land mask used. The bathymetries for \textbf{(a)} 5 Ma where the panama passage is closed linking the north and south American continents. \textbf{(b)} The present day situation.}
\end{figure}
\clearpage
\begin{multicols}{2}
