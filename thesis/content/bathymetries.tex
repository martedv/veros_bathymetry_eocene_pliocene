\subsection{Creating Bathymetries} \label{sec:bathys}
To facilitate the model a set of 14 bathymetries were created in $5$Ma time steps. These run from 65Ma to the present day configuration. These were reconstructed from bathymetries gained in \cite{Baatsen2016Aug}.
These bathymetrys were subsequently scaled to a 4 degree model and changed to address passage openings in the 4 degree case. Due to the low resolution of the model, choices have to be made with respect to the opening of certain passages. One of the choices that was made is that the northern Arctic sea is closed of in all of the bathymetrys. This is mainly due to the fact that 4 degree models do not have enough resolution to facilitate this sea and Veros lacking the ability to have polar flow.
The main events that shape the oceanic passages can be devided into time periods. These time periods are defined as follows in this paper. This deviates from their definitions in literature but serves only as a means of applying a name to the time steps.
\begin{table}[H]
	\begin{tabular}{lll}
		&From &Until \\
		Paleocene & 65Ma&55Ma    \\
		Eocene    & 50Ma&35Ma     \\
		Oligocene & 30Ma&20Ma    \\
		Miocene   & 15Ma&Present 
	\end{tabular}
\end{table}

\subsubsection{Paleocene}
In the paleocene a vast Pacific exists almost serving as a single basin. This period is largely characterized by the growth and development of a larger atlantic basin. Subsequently a decrease in size of the pacific basin is also seen. The drake passage is explicitly chosen to be closed in this time period, there is some evidence of it being opened in the paleocene due to a major change in the motion of the South American and Antarctic plates until about 50Ma (\cite{Livermore2005Jul}). However, the evidence proposes a shallow opening of less than 1 km in depth. These uncertainties and the extemely small nature of the basin has led to the decision to close the passage until its certain deep water connection starting after the late Eocene as indicated by \cite{Livermore2005Jul}.
\end{multicols}
%example full width overturning

\begin{figure}[H]
\includegraphics[width=\linewidth]{bathymetries_paleocene.pdf}
\caption{Paleocene Bathymetries showing \textbf{(a)} The bathymetry of 65 Ma. \textbf{(b)} The bathymetry of 60 Ma. \textbf{(c)} The bathymetry of 55 Ma}
\end{figure}

\begin{multicols}{2}
\subsubsection{Eocene}
The Eocene in contrast to the Paleocene is distinguished by The opening of certain passages connecting oceanic basins. These effects are often studied extensively for each individual passage in the literature. Choosing the exact timing for opening the passages is done manually by looking at often active research. The first of such passages to open is the Tasman passage which is opened at 35Ma as a shallow passage slowly growing in size(\cite{Lawver2003Sep}). The Tasman passage opening is believed to have had a large impact on the onset of the Atlantic circumpolar current (ACC).
From the onset of the early Eogene the Indian Continent has been fast moving towards the north slowly closing the northern passage between the Indian ocean and the Tethys seaway. The water passage is closed from 35Ma based on \cite{Najman2010Dec}.
\end{multicols}
%example full width overturning

\begin{figure}[H]
\includegraphics[width=\linewidth]{bathymetries_eocene.pdf}
\caption{Eocene Bathymetries showing \textbf{(a)} The bathymetry of 50 Ma. \textbf{(b)} The bathymetry of 45 Ma. \textbf{(c)} The bathymetry of 40 Ma.\textbf{(d)} The bathymetry of 35 Ma}
\end{figure}

\begin{multicols}{2}
\subsubsection{Oligocene}
From the onset of the Oligocene The Total circulation of water around the Antarctic basin is finalized by the opening of the shallow Drake passage at around 30Ma. 30Ma is specifically chosen to differentiate between the opening of the drake and Tasman passages. Especially since there is still some debate on the exact timing of drake passage opening (\cite{Scher2006Apr}; \cite{Livermore2005Jul}). These openings coincide with the onset of the Antarctic circumpolar current that has had major effects on the global climate variability. Furthermore, The Oligocene is characterized by the further expansion of the Atlantic basin and a Tethys seaway that is becoming more shallow. 
\end{multicols}
%example full width overturning

\begin{figure}[H]
\includegraphics[width=\linewidth]{bathymetries_oligocene.pdf}
\caption{Oligocene bathymetries showing: \textbf{(a)} The bathymetry of 30 Ma. \textbf{(b)} The bathymetry of 25 Ma. \textbf{(c)} The bathymetry of 20 Ma.}
\end{figure}

\begin{multicols}{2}
\subsubsection{Miocene}
The Miocene is Characterized by Some more passage closures. The Tethys seaway had been decreasing in size in the previous 20Ma. It finally fully detaches the mediteranian sea to the Indian ocean from 15Ma onward(\cite{Hamon2013Nov}). Then another major change occurs with the closure of the panama seaway from 5Ma onward  (\cite{Molnar2008Jun}; \cite{Pindell1988Dec}). Stopping the mid latitude throughflow between the Atlantic and Pacific basins.
\end{multicols}
%example full width overturning

\begin{figure}[H]
\includegraphics[width=\linewidth]{bathymetries_miocene.pdf}
\caption{Miocene bathymetries showing \textbf{(a)} The bathymetry of 15 Ma. \textbf{(b)} The bathymetry of 10 Ma. \textbf{(c)} The bathymetry of 5 Ma.\textbf{(d)} The present day bathymetry}
\end{figure}

\begin{multicols}{2}
