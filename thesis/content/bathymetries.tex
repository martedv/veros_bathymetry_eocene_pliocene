To facilitate the model a set of 14 bathymetries were created in $5$Ma time steps. These run from 65Ma to the present day configuration. These were reconstructed from bathymetries gained in \cite{Baatsen2016Aug}.
These bathymetrys were subsequently scaled to a 4 degree model and changed to address passage openings in the 4 degree case. Due to the low resolution of the model, choices have to be made with respect to the opening of certain passages. One of the choices that was made is that the northern Arctic sea is closed of in all of the bathymetrys. This is mainly due to the fact that 4 degree models do not have enough resolution to facilitate this sea and Veros lacking the ability to have polar flow.
The main events that shape the oceanic passages can be devided into time periods. These time periods are defined as follows in this paper. This deviates from their definitions in literature but serves only as a means of applying a name to the time steps.
\begin{table}[H]
	\begin{tabular}{lll}
		&From &Until \\
		Paleocene & 65Ma&55Ma    \\
		Eocene    & 50Ma&35Ma     \\
		Oligocene & 30Ma&20Ma    \\
		Miocene   & 20Ma&Present 
	\end{tabular}
\end{table}
%TODO (Relevant figures from paleoscene showing changes)
In the paleocene a vast Pacific exists almost serving as a single basin. This period is largely characterized by the growth and development of a larger atlantic basin. Serving to decrease the size of the pacific basin. We thus expect that the main flows here are concentrated in the pacific. There is probaby no possibility for northern sinking in the atlantic due to the smaller size of this basin.

The Eocene in contrast to the Paleocene is distinguished by The opening of certain passages connecting oceanic basins. These effects are often studied extensively for each individual passage in the literature. Choosing the exact timing for opening the passages is done manually by looking at often active research. This was done taking into account big uncertainties in the exact timing of the openings often making choises such that the diffirences in passage openings can be seen in the final integrations.
 
The first of such passage changes that occurs is the Indian continent colliding with the Eurasian continent. This has the effect of closing the deep water formations between the Thetys sea and the Pacific ocean. The first throughflow over the indian continent is possible at 45Ma.

Next the Tasman passage is opened at 35Ma as a shallow passage slowly growing in size(\cite{Lawver2003Sep}). The Tasman passage opening is believed to have had a large impact on the onset of the ACC. The Total circulation of water around the Antarctic basin is finalized by the opening of the shallow Drake passage at around 30Ma. 30Ma is specifically chosen to differentiate between the opening of the drake and Tasman passages. Especially since there is still some debate on the exact timing of drake passage opening (\cite{Scher2006Apr}). These openings will probably result in the onset of the Antarctic circumpolar current that has had major effects on the global climate variability.

%30ma drake passage opening (chosen as to diffirentiate between diffirent stages)
%35ma Indian passage closure
%35ma Tasman passage opening
%Indian continent colliding

%TODO Oligocene changes
The next time period is the Oligocene Which is largely characterized by the deepening of the Tasman and drake passage and further expansion of the Atlantic basin. Here we expect to see the onset of further strengthening of the ACC.

%TODO Miocene

The Miocene is Characterized by Some more passage closures. The Thetys seaway had been decreasing in size in the previous 20Ma. Also the Indonesian passage is significantly decreasing in size due to the onset  of multiple islands and the further displacement of the australian continent. This results in a narrower and shallower passage. Furthermore the Tethys seaway is finally closed at 15Ma(\cite{Hamon2013Nov}), Making northern flow over Africa between the Atlantic and Indian oceans impossible from this point onward. Then another major change occurs with the closure of the panama seaway (\cite{Molnar2008Jun}; \cite{Pindell1988Dec}). Stopping the mid latitude throughflow between the Atlantic and Pacific basins.

%0Ma current
%5Ma Wider Indonesia
%10ma last CAS
%14ma Thetys seaway last open 15ma first occurrence

