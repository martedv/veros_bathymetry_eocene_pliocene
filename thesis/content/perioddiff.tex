\subsection{Temperature and Salinity}\label{sec:temp+sal}
To get a better picture of the changes that occur during the time periods the differences in sea temperature and salinity between the models and the original begin conditions are compared. The sea surface temperature and salinity are not used here because of the restoring boundary conditions. Instead the results at a depth of $-245m$ are used. It is important to note that these restoring boundary conditions with the simplified forcings make any discussion on the thermohaline circulation to be of limited value.

\subsubsection{Salinity Changes}
First, looking at the salinity profiles in \cref{fig:sss_total} the gyres seen in the BSF are clearly visible(\cref{sec:BSF}).
Little other changes can be deduced. Especially in the pacific where only marginal changes are observed. However, the closure of the Thetys Seaway can be seen in both the Atlantic and Indian basin. Fresher water from the arctic is transported south when the Thetys transport is low and the northern subtropical gyre is started. Also a variability in the Indian ocean is seen which coincides with the variability in the Thetys Seaway throughflow. Indicating that a connection between these exists, allowing salt water to flow away from the Indian basin. Also the Eocene-Oligocene boundary between 35 and 30Ma shows a change in salinity in the southern subpolar region. A connection between the subpolar gyres seen in the BSF due to the opening of the Tasman and Drake passage is clear here. Making the Antarctic region slightly more saline. 

\subsubsection{Temperature Changes}
To look at the temperature changes we look at the profiles compared to the initial conditions (\cref{fig:sst_total}). Again the major ocean gyres are visible. Here the strong influence of the restoring boundary conditions is visible. Similar to what was observed in the Salinity profiles there is a variability observed in the Eocene-Oligocene boundary. The ACC-like cell described in \cref{sec:eocenemoc} seems to keep the warm water south. After 35Ma, when the ACC is fully developed the temperature stabilizes again.

In the Indian ocean the influence of the Indian continent's movements is seen. The Indian ocean being a few degrees cooler before the Eocene-Oligocene boundary when India is still detached from Eurasia. Two processes might explain this. First of all, the decrease in throughflow through the Indonesian passage and second of all the reduction of the blocking effect of the Indian continent of the subtropical gyre.