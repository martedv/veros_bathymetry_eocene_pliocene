\subsection{Barotropic Stream function}
\label{sec:BSF}


Next, we will look at the barotropic stream function for each of the time steps discussed in this paper. Some of the flows that are discussed in this section are closely related to the flows explained in \cref{sec:throughflow}. Here we will have a stronger focus on the gyres seen in the ocean and their relative strength in a time sense. Each of the oceanic basins is discussed in detail. An overview of each of the barotropic stream functions can be seen in \cref{fig:bsf_total}. The boundary values of the BSF are not shown here. This is due to the previously stated fact that they are excluded from the model output produced by Veros. It must however be noted that this does not mean that flows through the passages are not modeled. In this case, the barotropic stream function serves only to see the major ocean gyres and how water is transported in these gyres.

\subsubsection{Indian Ocean}
The Indian ocean and especially the Indian Continent moving northward seems to be one of the most interesting artifacts of these simulations. When the Indian continent is still within the subtropical gyre range in the early Paleocene. We see that it has a large blocking effect on the subtropical gyre in the Indian ocean. We also see, as observed in the Flow patterns for each of the basins, a change from current moving north over India to moving east and then up towards the Atlantic basin. Something that is similarly observed in \cite{omta2003physical}. However as noted in \cref{sec:throughflowp} we do not observe a the often shown circum-India current (\cite{omta2003physical};
\cite{von2006effect}). In the Paleocene, the Indian continent seems to be the most influential in establishing the ocean gyres. The stark contrast between 65 and 60Ma BSF can be explained due to the Kohistan-Ladakh Arc island ridge north of India. These islands block the flow from developing a strong current around the Indian continent. 


\subsubsection{Pacific Ocean}
The Pacific Ocean is of particular interest in this case. One of the main things that we see is a large variability in the strength of the southern subtropical gyre. There appears to be a variability in the order of $\pm 20 Sv$.  The variability is most pronounced when the ACC is not yet fully developed and is most pronounced in the Paleocene and early Eocene. The transport is particularly extreme at times where the Thetys throughflow is the larger. The size of the southern subtropical gyre seems to relate to the Thetys values seen in \prettyref{fig:throughflow}. Here we see a current around the planet through the Thetys, Indonesian, and Panama passages. This can explain why such a large positive stream function can be seen in the southern Pacific.

\subsubsection{Atlantic Ocean}
The Atlantic Ocean seems to be the least influential in most of the timesteps studied here. This is in large part thanks to the fact that the Atlantic basin is relatively small in the Paleocene growing slowly over time. The one gyre of note in the Atlantic which exists until the onset of the ACC is the southern subpolar gyre. This gyre shows similar variability to the Gyres in the Pacific and Indian oceans. The onset of the ACC also seems to coincide with the growth of the southern subtropical gyre. Also, the northern subpolar gyre is hardly visible here at all. This is likely due to low resolution used by this model not being able to have proper in and outflow of the northern arctic sea.


%The detachment of Australia to the antarctic current at 35Ma introduces a strong flow ($17Sv$) through the newly formed Tasman passage. However there is no