\subsection{Barotropic Stream function}
\label{sec:BSF}

Next we will look at the barotropic stream function for each of the time steps discussed here. Some of the flows that are discussed in this section are closely related to the flows explained in \fref{sec:throughflow}. Here we will have a stronger focus on the flows gyres seen in the ocean and their relative strength in a time sense. Each of the oceanic basins is discussed in detail. An overview of each of the barotropic stream functions can be seen in \fref{fig:bsf_total}. It is very visible that the boundary conditions of the BSF are not shown here. This is due to the previously stated fact that they are excluded from the model output produced by Veros. It must however be noted that this does not mean that flows through the passages are not modeled but rather only that the passages themselves do not show up on the plots of the barotropic stream function. In this case the barotropic stream function serves only to see the major ocean gyres and how water is transported in these gyres.

\subsubsection{Indian Ocean}

\subsubsection{Pacific Ocean}
The pacific ocean is of particular interest in this case. It is easily vi
\subsubsection{Atlantic Ocean}



%The detachment of Australia to the antarctic current at 35Ma introduces a strong flow ($17Sv$) through the newly formed Tasman passage. However there is no