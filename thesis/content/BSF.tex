\subsection{Barotropic Stream function}
\label{sec:BSF}

Next we will look at the barotropic stream function for each of the time steps discussed here. Some of the flows that are discussed in this section are closely related to the flows explained in \prettyref{sec:throughflow}. Here we will have a stronger focus on the flows gyres seen in the ocean and their relative strength in a time sense. Each of the oceanic basins is discussed in detail. An overview of each of the barotropic stream functions can be seen in \prettyref{fig:bsf_total}. It is very visible that the boundary conditions of the BSF are not shown here. This is due to the previously stated fact that they are excluded from the model output produced by Veros. It must however be noted that this does not mean that flows through the passages are not modeled but rather only that the passages themselves do not show up on the plots of the barotropic stream function. In this case the barotropic stream function serves only to see the major ocean gyres and how water is transported in these gyres.

\subsubsection{Indian Ocean}
The Indian ocean and especially the indian Continent moving northward seems to be one of the most interesting artifacts of these simulations. When the indian continent is still within the subtropical gyre range in the early Paleocene. We see that it has a large blocking effect on the Subtropical gyre in the Indian ocean. We also see, as observed in the Flow patterns for each of the basins, a change from current moving north over india to moving east and then up towards the Atlantic basin.
\subsubsection{Pacific Ocean}
The pacific ocean is of particular interest in this case. One of the main things that we see is a large fluctuation in the strength of the southern subtropical gyre. This fluctuation is a diffirence of $\pm 20 Sv$.  Especially when the ACC is not yet developed. This is especially visible in the Paleocene and early Eocene where the transport is particularly extreme at places where the Thetys throughflow is the largest. The size of the southern subtropical gyre seems to relate to the Thetys values seen in \prettyref{fig:throughflow}. Here we see a round earth current through the Thetys, Indonesian and Panama passages exists. This can explain why such a largely positive streamfunction can be seen in the southern pacific. This  Where this only changes with the onset of the ACC.
\subsubsection{Atlantic Ocean}
The Atlantic basin seems to be the most quiet basin here. This is in large part thanks to the fact that the atlantic basin is so small in the beginning of our time series. One of the flows that is of particular interest here is the subpolar gyre that exists the entire time until the onset of the ACC where it is replaced. The onset of the ACC also seems to coincide with the growth of the southern subtropical gyre. Also the northern subpolar gyre is hardly visible here at all. This is likely due to low resolution used by this model not being able to have proper in and outflow of the arctic sea here. 


%The detachment of Australia to the antarctic current at 35Ma introduces a strong flow ($17Sv$) through the newly formed Tasman passage. However there is no