The geometry and resulting bathymetry of our planet is an ever changing phenomenon. In the last 120 Ma the earth moved from having one major oceanic system in the Pacific with a single large continent to the current 3 ocean system (\cite{besse2002apparent}). The Bathymetry changes that occurred in this time period are characterized by the opening and closing of certain passages through which exchange of water between the oceanic basins is observed. The exact timing of passage openings is a topic of rigorous debate in literature (\cite{Scher2006Apr}, \cite{Schmidt2007Jan}).


One of the changes on which there is general consensus, is the inception and expansion of the Atlantic ocean and the resulting decrease in size of the Pacific basin. The creation of the Atlantic basin has had major effects on the earth's climate, especially resulting in massive localized changes such as the temperate European climate, due to the north Atlantic meridional overturning current (AMOC). This creates the current Northern sinking oceanic throughflow in the Atlantic. However it is unknown when exactly this northern sinking started. With the past non-existance of the Atlantic it must have started some time in the last 40Ma with the advent of a larger Atlantic (\cite{Abelson2017onset}). 

The result of these bathymetry changes on the oceanic stream function and the resulting overturning currents is something that has been previously studied by \cite{Mulder2017Jul}. They however found that using a Jacobian matrix for continuation in each of the model years fails to simulate the onset of the Northern sinking AMOC that is physically observed. Here we instead propose to use a general circulation ocean model with only a changing bathymetry keeping the same initial forcing for each time step. This eliminates the need for a continuation using the Jacobian matrix method proposed in \cite{Mulder2017Jul}.

This paper will focus solely on changes in bathymetry using simplified zonally averaged global forcings. The results of the model will be used to estimate global changes in oceanic through flow at the critical passages. Furthermore the strength of the meridional overturning currents (MOC) will be studied.
