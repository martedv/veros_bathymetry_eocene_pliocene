The geography and bathymetry (depth profile of the ocean) of our planet is an ever-changing phenomenon. In the last 120 Ma (million years), the earth moved from having one major oceanic system in the Pacific, to the current 3 ocean system (\cite{besse2002apparent}). The Bathymetry changes that occurred in this period are characterized by the opening and closing of certain passages through which exchange of water between oceanic basins is observed. These passage changes are a vital part of understanding the global thermohaline circulation.  The exact timing of passage openings is a topic of rigorous debate in literature (\cite{Scher2006Apr}, \cite{Schmidt2007Jan}).

However, one of the changes on which there is a consensus; is the inception and expansion of the Atlantic ocean. This expansion results in a decrease in the size of the Pacific basin. The creation of the Atlantic basin has had major effects on the earth's climate, resulting in massive localized changes such as the temperate European climate. The North Atlantic meridional overturning circulation (AMOC) is now understood to be essential to the present-day thermohaline circulation. The AMOC is the result of a deep water formation in the North Atlantic. In the northern Atlantic, there is a northward flow of water with substantial heat energy. This water sinks when it reaches the arctic waters. This is because the flow rapidly loses its heat energy due to the large temperature gradient in the arctic. The loss of energy causes an increase in density and subsequent sinking of the flow. This flow is often called the "ocean Conveyer belt", a term first coined by \cite{broecker1991great}. However, it is unknown when exactly this northern sinking started. With the past nonexistence of the Antarctic Circumpolar Current (ACC) and the relatively small size of the Atlantic ocean, the AMOC must have seen its inception sometime in the last 40Ma (\cite{Abelson2017onset}). 

The result of these bathymetry changes on the oceanic stream function and the resulting overturning currents are something that have been previously studied by \cite{Mulder2017Jul}. However, they found that using a 3D model for different geographies in each of the model years failed to simulate the onset of the Northern sinking AMOC. They used a continuation method for the forcing of each timestep, taking the previous timestep as a reference. In this paper we propose to use a similar general ocean circulation model (GCM) with only a changing bathymetry and highly simplified zonal forcings. To accomplish this we use the relatively young GCM Veros.

This paper will focus on the effect of changes in bathymetry when using highly simplified zonally averaged forcings for the last 65Ma. The results of the model will be used to estimate global changes in oceanic throughflow at oceanic passages. Furthermore, the strength of the meridional overturning currents (MOC) and the thermohaline circulation will be studied.

