\section{Summary}
In this paper, we have presented a simplified approach to the modeling of past climate systems using Veros. This paper focused heavily on simplified forcings of the global oceanic basins. This allows us to efficiently look at the effect of changes in geometry on the major oceanic flows. The results shown here are of relatively low resolution and highly idealized boundary conditions. But they still manage to capture some of the features of more complex models. The integrations were done on a consumer computer showing that it is possible to do even larger ocean simulation research on readily available hardware.

In the barotropic stream functions, we observed high variability between the different integrations. Especially in the Paleocene where it was mainly attributed to the large variation in the Indian ocean. It thus appears that the exact location of the Indian continent may have had profound implications on the past oceanic circulations. In our study of the passage throughflow, we find that there is a flow reversal in the Panama passage after the closure of the Tethys Seaway. Here we also find large variability in the passage throughflows before the onset of the Antarctic Circumpolar Current. Furthermore, we observe flow reversal in the Aghulas passage between the Indian and Atlantic oceans coinciding with to the opening of the drake passage. 

We were unable to accurately predict the meridional overturning circulation. We do however find some evidence for a total absence of transport over the equator before the onset of the ACC. Also the Deacon cell was seen with the onset of the ACC which is an artifact of the limited depth resolution used for our model.

Our analysis of the temperature and salinity profiles has shown that the onset of the ACC caused a large shift in global temperatures. The Eocene-Oligocene boundary in particular shows variability caused by the onset of the "proto-ACC" and actual ACC with a shallow Drake passage. Also, variability in the salinity profiles showed that the during the northward migration of the Indian Continent the blocking effect of this continent might have had a cooling-effect on the entire northern Indian ocean and Thetys seaway.

In general, the wind-driven circulation we observe quite accurately manage to capture the changes in volume transport through the major passages. We find that using zonally averaged forcings does have implications on the strength of the flows but the large scale changes are captured. We can conclude that the simulations were quite successful in capturing the wind-driven circulation but require much higher resolution to be useful in present-day research.
