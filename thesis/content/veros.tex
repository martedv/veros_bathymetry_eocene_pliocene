Veros is a flexible ocean model specifically designed for studying simple problems in oceanography. It is a strictly Ocean only model based on the succesful PyOm2 model. It was designed from the ground up with flexibility in mind. Written entirely in python. A language that is now widely deployed and taught too many students unlike it's Fortran predecessor. This allows easy editing of the code running the ocean model itself during the research phase. Cutting valuable time spent on figuring out the often cumbersome Fortran models of the past. Veros is specifically well suited for researching the effect of changes in both forcings and bathymetrys. They can be easily edited using Python. These features in particular are heavily used in this paper. One of the most extensively used features for example is the fact that any bathymetry can without further manual specifications of islands be used for stream function calculation.

%TODO Mathematics behind the model. Continouation mechanics.


Ocean modeling has been an area of continued progress. The resolutions of the models have been steadily increasing since the inception of the first computerized ocean models. However, due to the age of some of these models and the continued adaptation of often old legacy Fortran code, many models have become enormous hurdles to get started with often resulting in frustration. The Veros\cite{Hafner2018Aug} ocean model project is trying to tackle this problem with a totally new code base written entirely in Python. Veros allows easy editing of forcing and geometry input and infinite flexibility in the model's setup without the hassle of learning Fortran.