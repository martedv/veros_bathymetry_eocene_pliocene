\subsection{Veros}
Veros is an ocean general circulation model (GCM) based on the successful pyOM2 model (\cite{Hafner2018Aug}). It was designed from the ground up with flexibility in mind. This flexibility cuts valuable time spent on figuring out the often cumbersome Fortran models of the past. Veros is specifically well suited for researching the effect of changes in both forcings and bathymetries. They can be easily edited using Python. These features in particular are heavily used in this paper. One of the most extensively used attributes is the fact that any bathymetry can, without further manual specifications, be used for stream function calculation.
The fact that Veros is fully written in Python helps as Python is a far more widespread language than Fortran and it is thus much easier to teach Veros to new students. 

Veros uses an Arakawa C-grid for it's calculations of finite differences. Here Veros uses a separate grid for tracers (temperature, salinity and density) than the grid used for vertical fluxes (velocities). For a more specific overview of the functionality the reader is referred to the \cite{BibEntry2020Jun}.

In this case, the models used are run on an 8 core (16 threads) machine using an MPI CPU configuration of 1 node. This is sufficient for the lower resolution models used in this paper. But it is noted that Veros allows the usage of multiple nodes to do calculations on much higher resolution problems.

