\subsection{Veros}
Ocean modeling has been an area of continued progress. The resolutions of the models have been steadily increasing since the inception of the first computerized ocean models. However, due to the age of some of these models and the continued adaptation of often old legacy Fortran code, many models have become enormous hurdles to get started with often resulting in frustration. The Veros ocean model project is trying to tackle this problem with a totally new code base written entirely in Python (\cite{Hafner2018Aug}). 
Veros is an ocean general circulation model (GCM) based on the successful PyOm2 model. It was designed from the ground up with flexibility in mind. This flexibility cuts valuable time spent on figuring out the often cumbersome Fortran models of the past. Veros is specifically well suited for researching the effect of changes in both forcings and bathymetrys (depth profiles of the oceans). They can be easily edited using Python. These features in particular are heavily used in this paper. One of the most extensively used features for example is the fact that any bathymetry can, without further manual specifications, be used for stream function calculation.
The fact that it is fully written in Python is especially useful as python is far more widespread than Fortran and it is thus much easier to teach Veros to new students. 

In this case the models used in this paper are run on an 8 core (16 threads) machine using an MPI CPU configuration of 1 node. This is sufficient for the lower resolution models used in this paper. But Veros allows the usage of multiple nodes to do calculations on much higher resolution problems.

