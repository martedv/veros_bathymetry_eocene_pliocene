Ocean modeling has been an area of continued progress. The resolutions of the models have been steadily increasing since the inception of the first computerized ocean models. However, due to the age of some of these models and the continued adaptation of often old legacy Fortran code, many models have become enormous hurdles to get started with often resulting in frustration. The Veros ocean model project is trying to tackle this problem with a totally new code base written entirely in Python (\cite{Hafner2018Aug}). 
Veros is a General Circulation Ocean model based on the successful PyOm2 model. It was designed from the ground up with flexibility in mind. This flexibility cuts valuable time spent on figuring out the often cumbersome Fortran models of the past. Veros is specifically well suited for researching the effect of changes in both forcings and bathymetrys. They can be easily edited using Python. These features in particular are heavily used in this paper. One of the most extensively used features for example is the fact that any bathymetry can, without further manual specifications of islands be used for stream function calculation.

In this case the models used in this paper were run on a 8 core (16 threads) machine using an MPI CPU configuration of 1 node. The Bohrium GPU possibilities of Veros were also tried but failed to result in much improvement in speed. Figure (figure on speed) shows the model speed of the integration. The total time needed to run all of the models was approximately one week.

(some more on how fast the model is)